% LaTeX source of my resume
% =========================

% Commented for easy reuse... ;)

% See the `README.md` file for more info.

% This file is licensed under the CC-NC-ND Creative Commons license.


% Start a document with the here given default font size and paper size.
\documentclass[10.5pt,letterpaper]{article}

% Set the page margins.
\usepackage[letterpaper,margin=0.75in]{geometry}

% Setup the language.
%\usepackage[english]{babel}
%\hyphenation{Some-long-word}

% Makes resume-specific commands available.
\usepackage{resume}

\renewcommand{\baselinestretch}{1.0}
\setlength{\parskip}{0em}

\usepackage{enumitem}
\setlist[itemize]{leftmargin=*}

\begin{document}  % begin the content of the document
\sloppy  % this to relax whitespacing in favour of straight margins

% title on top of the document
\maintitle{Jeff Irwin}{}{Last updated on \today}

\nobreakvspace{0.3em}  % add some page break averse vertical spacing

% \noindent prevents paragraph's first lines from indenting
% \mbox is used to obfuscate the email address
% \sbull is a spaced bullet
% \href well..
% \\ breaks the line into a new paragraph

\noindent
\href{mailto:ENV\_EMAIL}{ENV\_EMAIL}\sbull
ENV\_MOBILE\sbull
\href{https://github.com/JeffIrwin}{github.com/JeffIrwin}
%\sbull\href{https://www.linkedin.com/in/jeff-irwin-54967332/}{https://www.linkedin.com/in/jeff-irwin-54967332/}

\noindent
ENV\_ADDRESSL1\sbull
ENV\_ADDRESSL2\sbull
ENV\_ADDRESSL3

\spacedhrule{1.0em}{-0.5em}  % a horizontal line with some vertical spacing before and after
\roottitle{Education}

\noindent \href{https://www.psu.edu/}{The Pennsylvania State University} \hfill \emph{May 2015} \break
\textbf{Master of Science in Mechanical Engineering} \hfill Cumulative GPA:  3.89 / 4.00

\vspace{1em}
\noindent \href{https://www.psu.edu/}{The Pennsylvania State University} \hfill \emph{May 2013} \break
\textbf{Bachelor of Science in Mechanical Engineering} \hfill Cumulative GPA:  3.94 / 4.00

\roottitle{Relevant Courses}
Non-Linear FEA, Thermodynamics of Materials, Solid Mechanics, Numerical Methods in Fluid Dynamics

\roottitle{Computer Skills}
Fortran, C++, Git, Bash, MATLAB, Java, Python, Abaqus, SolidWorks, Windows and Linux OS

\spacedhrule{0.5em}{-0.5em}
\roottitle{Experience}

\noindent \textbf{Lead Software Developer} \hfill 2015 -- 2016 \break
\href{http://www.pancomputing.com}{Pan Computing} \hfill \textsc{State College, PA}
\begin{itemize}
	\setlength{\parskip}{0em}
	\item Designed databases and implemented encryption algorithms to store and protect proprietary data
	\item Administered file servers for end-users to download and install CUBES\textsuperscript\textregistered\ software
	\item Ported CUBES\textsuperscript\textregistered\ command line interface to Windows OS and developed a graphical user interface
	\item Integrated CUBES\textsuperscript\textregistered\ with Reprise License Manager for floating licensing
	\item Developed a geometric compensation technique to use the results of CUBES\textsuperscript\textregistered\ to reduce distortion
	\item Verified and validated CUBES\textsuperscript\textregistered\ and assisted end-users
\end{itemize}

\noindent \textbf{Consultant} \hfill 2015 -- 2016 \break
\href{http://www.pancomputing.com}{Pan Computing} \hfill \textsc{State College, PA}
\begin{itemize}
	\setlength{\parskip}{0em}
	\item Tested CUBES\textsuperscript\textregistered\ software and wrote examples manuals
\end{itemize}

\noindent \textbf{Research Assistant} \hfill 2012 -- 2015 \break
\href{https://www.psu.edu}{The Pennsylvania State University} \hfill \textsc{State College, PA}
\begin{itemize}
	\setlength{\parskip}{0em}
	\item Used the finite element method to model additive manufacturing and compared results to experimental data
	\item Developed a new model of laser heat input to significantly speed up simulations
	\item Wrote Fortran and C++ codes for automatic mesh generation and adaptive mesh refinement
	\item Implemented algorithms to simulate microstructural evolution in Ti-6Al-4V during additive manufacturing
	\item Wrote MATLAB and Fortran codes for pre-processing and post-processing of simulation data
	\item Worked with Dr.\ Pan Michaleris and other graduate and undergraduate researchers
\end{itemize}

\spacedhrule{0.5em}{-0.5em}
\roottitle{Journal Publications}

\begin{enumerate}[leftmargin=*]
	\setlength{\parskip}{0em}
	\item Denlinger, Erik R., Jeff Irwin, and Pan Michaleris. ``Thermomechanical modeling of additive manufacturing large parts.'' \emph{Journal of Manufacturing Science and Engineering} 136.6 (2014): 061007.
	\item Kriczky, Dennis A., Jeff Irwin, Edward W. Reutzel, Pan Michaleris, Abdalla R. Nassar, and James Craig. ``3D spatial reconstruction of thermal characteristics in directed energy deposition through optical thermal imaging.'' \emph{Journal of Materials Processing Technology} 221 (2015): 172-186.
	\item Irwin, Jeff, and Pan Michaleris. ``A Line Heat Input Model for Additive Manufacturing.'' \emph{Journal of Manufacturing Science and Engineering} 138.11 (2016): 111004.
	\item Irwin, Jeff, Edward W. Reutzel, Pan Michaleris, Jay Keist, and Abdalla R. Nassar. ``Predicting Microstructure from Thermal History during Additive Manufacturing for Ti-6Al-4V.'' \emph{Journal of Manufacturing Science and Engineering} 138.11 (2016): 111007.
\end{enumerate}


\spacedhrule{0.5em}{-0.5em}
\roottitle{Conference Publications}
\begin{enumerate}[leftmargin=*]
	\setlength{\parskip}{0em}
	\item J. Irwin and P. Michaleris. ``A Line Heat Input Model for Additive Manufacturing.'' Manufacturing Science and Engineering Conference. Hilton Charlotte University Place, Charlotte, NC. 10 June 2015.
	\item G. Loughnane, S. Kuntz, N. Klingbeil, J. Sosa, J. Irwin, A. Nassar, and E. Reutzel. ``Application of a microstructural characterization uncertainty quantification framework to Widmanst\"atten $\alpha$-laths in additive manufactured Ti-6Al-4V.'' Solid Freeform Fabrication Symposium. AT\&T Executive Education and Conference Center, Austin TX. 10 August 2015.
\end{enumerate}

\roottitle{Conference Abstracts}
\begin{enumerate}[leftmargin=*]
	\setlength{\parskip}{0em}
	\item J. Irwin and P. Michaleris. ``A Line Heat Input Model for Additive Manufacturing.'' Materials Science \& Technology. David L. Lawrence Convention Center, Pittsburgh, PA. 14 October 2014.
	\item J. Irwin, E. Reutzel, P. Michaleris, J. Keist, A. Nassar, and G. Loughnane. ``Linking Thermal History to Microstructure during AM of Ti-6Al-4V.'' Solid Freeform Fabrication Symposium. AT\&T Executive Education and Conference Center, Austin TX. 10 August 2015.
	\item 1 A. Dunbar, E. Denlinger, P. Michaleris, P. Guerrier, R. Buchannan, J. Heigel, J. Jones. ``In Situ Temperature and Distortion Measurements for PBF Model Validation.'' Solid Freeform Fabrication Symposium. AT\&T Executive Education and Conference Center, Austin TX. 10 August 2015.
	\item J. Irwin, E. Reutzel, P. Michaleris, J. Keist, A. Nassar, and G. Loughnane. ``Linking Thermal History to Microstructure during AM of Ti-6Al-4V.'' Materials Science \& Technology. Greater Columbus Convention Center, Columbus, OH. 5 October 2015.
	\item \footnote{Presented by J. Irwin} A. Dunbar, E. Denlinger, P. Michaleris, P. Guerrier, R. Buchannan, J. Heigel, J. Jones. ``In Situ Temperature and Distortion Measurements for Powder-Bed Fusion Model Validation.'' Materials Science \& Technology. Greater Columbus Convention Center, Columbus, OH. 5 October 2015.
	\item D. Corbin, J. Irwin, A. Nassar, E. Reutzel, and P. Michaleris. ``Temperature Calibration of an Additive Manufacturing Process-Monitoring Thermal Imaging System.'' Materials Science \& Technology. Greater Columbus Convention Center, Columbus, OH. 5 October 2015.
\end{enumerate}

\spacedhrule{0.5em}{-0.5em}
\roottitle{Activities and Awards}

\end{document}

